\documentclass[12pt]{article}
\usepackage[utf8]{inputenc}

\usepackage{titlesec}
\newcommand{\sectionbreak}{\clearpage}

\usepackage{natbib}
\usepackage{graphicx}
\usepackage{imakeidx}
\usepackage{caption}
\usepackage{listings}

\usepackage{hyperref}
\hypersetup{
    colorlinks=true,
    linkcolor=blue,
    filecolor=magenta,      
    urlcolor=cyan,
}

%%%%%%%%%%%%%%%%%%%%%%%%%%%
%%        EDICION        %%
%%%%%%%%%%%%%%%%%%%%%%%%%%%

\usepackage{todonotes}

%%%%%%%%%%%%%%%%%%%%%%%%%%%

\graphicspath{ {images/} }
\makeindex


% Title data
\title{Control parental de accesos WiFi}
\author{Esteban Omelio Puentes Silveira}
\date{Mayo, 2017}

\begin{document}
\maketitle

%\begin{figure}[h!]
%    \centering
%        \includegraphics[scale=0.2]{logo_uvigo.png}
%        \caption*{Universidad de Vigo}
%        \label{fig:uvigo}
%\end{figure}

\clearpage


%%%%%%%%%%%%%%%%%%%%%%%%%%%%%%%%%%%
%% ÍNDICE
%%%%%%%%%%%%%%%%%%%%%%%%%%%%%%%%%%%

\renewcommand\contentsname{Índice}
\tableofcontents
\printindex


%%%%%%%%%%%%%%%%%%%%%%%%%%%%%%%%%%%
%% CONTENIDO
%%%%%%%%%%%%%%%%%%%%%%%%%%%%%%%%%%%
\section{Introducción}
%haberá que incluír unha introdución ao problema e xustificación do
%traballo realizado. En caso de que o TFG integre ou desenvolva traballos feitos na
%actividade doutras materias da titulación, o/a estudante deberá especificar os devanditos
%traballos e materias nesta sección.*/

Internet, desde su creación, ha sido un medio cada vez más accesible; actualmente cuenta con 3,731,973,423 de usuarios, cifra que representa casi el 50\% de los habitantes del planeta. 

El uso de redes Wi-Fi ha permitido que los usuarios tengan una mayor disponibilidad de la conexión a Internet llevándola a sus dispositivos móviles, tanto en casa como fuera de ella. Actualmente es común que los menores de edad dispongan de dispositivos móviles desde los cuales puedan acceder a Internet, sin embargo, esto conlleva a que debido a su inmadurez y falta de experiencia, terminen pasando más tiempo del recomendado frente a una pantalla, sin dedicarle el tiempo necesario a sus obligaciones, desatendiendo sus estudios o directamente perdiendo la comunicación con los seres que lo rodean. Los padres de los menores normalmente toman medidas para prevenir 


\bibliographystyle{plain}
\bibliography{references}

\end{document}
